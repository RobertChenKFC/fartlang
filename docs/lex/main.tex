\documentclass[titlepage]{article}
\usepackage{amssymb}
\usepackage[
	margin=1.5cm,
	includefoot,
	footskip=30pt,
]{geometry}

\usepackage{amsthm}
\usepackage[open,openlevel=0]{bookmark}
\usepackage{mathrsfs}
\usepackage{dsfont}
\usepackage{amsmath}
\usepackage[dvipsnames]{xcolor}
\usepackage{tikz}
\usepackage{multirow}
\usepackage{CJKutf8}
\usepackage{enumerate}
\usepackage{minted}
\usepackage{listings}
\usepackage{xcolor}
\usepackage{wasysym}
\usepackage{enumerate}
\usepackage[RPvoltages]{circuitikz}
\usepackage{soul}
\usepackage{karnaugh-map}
\usepackage{booktabs}
\usepackage{subcaption}
\usepackage{algorithm}
\usepackage{algpseudocode}
\newcommand{\pushcode}[1][1]{\hskip\dimexpr#1\algorithmicindent\relax}
\newcommand{\MyComment}[2][.5] {\Comment{\parbox[t]{#1\linewidth}{#2}}}
\makeatletter

\newenvironment{breakablealgorithm}
  {% \begin{breakablealgorithm}
   \begin{center}
     \refstepcounter{algorithm}% New algorithm
     \hrule height.8pt depth0pt \kern2pt% \@fs@pre for \@fs@ruled
     \renewcommand{\caption}[2][\relax]{% Make a new \caption
       {\raggedright\textbf{\ALG@name~\thealgorithm} ##2\par}%
       \ifx\relax##1\relax % #1 is \relax
         \addcontentsline{loa}{algorithm}{\protect\numberline{\thealgorithm}##2}%
       \else % #1 is not \relax
         \addcontentsline{loa}{algorithm}{\protect\numberline{\thealgorithm}##1}%
       \fi
       \kern2pt\hrule\kern2pt
     }
  }{% \end{breakablealgorithm}
     \kern2pt\hrule\relax% \@fs@post for \@fs@ruled
   \end{center}
  }
\makeatother

\usetikzlibrary{arrows, shapes.gates.logic.US, calc, positioning}

\theoremstyle{plain}
\newtheorem*{axiom}{Axiom}
\newtheorem*{theorem}{Theorem}
\newtheorem*{observation}{Observation}
\newtheorem*{exercise}{Exercise}

\theoremstyle{definition}
\newtheorem*{definition}{Definition}

\theoremstyle{remark}
\newtheorem*{solution}{Solution}
\newtheorem*{remark}{Remark}

\DeclareMathOperator{\dom}{dom}

\title{Lexical Analysis}
\author{Robert Chen}

\begin{document}

\maketitle

\newpage
\tableofcontents

\newpage
\section{Regular Expressions}
The following is the definition of a \textit{regular expression}:
\begin{definition}
  A \textit{regular expression} (\textit{regex} for short) over finite alphabet
  $\Sigma$ is recursively built from the following rules:
  \begin{itemize}
    \item Any $a\in\Sigma$ is a regex.
    \item If $\alpha,\beta$ are regex, then $\alpha\beta$ is a regex.
    \item If $\alpha,\beta$ are regex, then $\alpha|\beta$ is a regex.
    \item If $\alpha$ is a regex, then $\alpha^*$ is a regex.
  \end{itemize}
\end{definition}
A regex defines a language --- a set of strings it accepts. The
following is the definiion of the language of a regex:
\begin{definition}
  Let $\gamma$ be a regex over $\Sigma$ and $L(\gamma)$ denote the language it
  accepts, then
  \begin{itemize}
    \item If $\gamma=\varnothing$, then
      \[
        L(\gamma)=\varnothing
      \]
    \item If $\gamma=a\in\Sigma$, then
      \[
        L(\gamma)=\{a\}
      \]
    \item If $\gamma=\alpha\beta$, then
      \[
        L(\gamma)=\{w_1\mathbin\Vert w_2:w_1\in L(\alpha),w_2\in L(\beta)\}
      \]
      where $\Vert$ denotes the concatenation operation on two strings.
    \item If $\gamma=\alpha|\beta$, then
      \[
        L(\gamma)=L(\alpha)\cup L(\beta)
      \]
    \item If $\gamma=\alpha^*$, then
      \[
        L(\gamma)=\bigcup_{w\in L(\alpha)}\bigcup_{n\in\mathds{N}}w^n
      \]
      where $w^n$ denotes $n$ concatenations of the same string $w$. Note that
      $w^0$ denotes the empty string (string of length 0), and is commonly
      denoted as $\varepsilon$.
  \end{itemize}
\end{definition}
We use regexs as a tool to describe what each type of token should look like,
and the language it defines a set containing all valid tokens of that type.
For common use, regexes are often defined over the ASCII characters and extended
with the following:
\begin{itemize}
  \item $x_a-x_b\equiv x_a|x_{a+1}|\cdots|x_b$: a range (can only be used in character
    classes), where $x_i$ represents the $i$-th ASCII character.
  \item $[a_1a_2\cdots a_n]\equiv a_1|a_2|\cdots|a_n$: a character class, where
    each $a_i$ is either from $\Sigma$ or a range.
  \item $[{}^\wedge x]\equiv [a_1a_2\cdots a_n]$: a negated character class, where
    $\{a_1,a_2,\ldots,a_n\}=\Sigma\setminus L([x])$.
  \item $.=[x_0-x_{127}]$: match any ASCII character.
  \item $\text{\textbackslash}d\equiv [0-9]$: digits.
  \item $\text{\textbackslash}w\equiv [A-Za-z]$: letters.
  \item $\alpha?\equiv \alpha|\varnothing^*$: 0 or 1 occurrences of $\alpha$.
  \item $\alpha+\equiv \alpha\alpha^*$: 1 or more occurrences of $\alpha$.
\end{itemize}

\section{Non-deterministic Finite Automata}
A \textit{non-deterministic finite automata} is defined as the following:
\begin{definition}
  A \textit{non-deterministic finite automata} (\textit{NFA} for short) is
  a 5-tuple $\mathcal{A}=(\Sigma,Q,q_0,F,\delta)$ where
  \begin{itemize}
    \item $\Sigma$ is a finite alphabet.
    \item $Q$ is a set of states.
    \item $q_0\in Q$ is the initial state.
    \item $F\subseteq Q$ is a set of accepting states.
    \item $\delta\subseteq (Q\times(\Sigma\cup\{\varepsilon\}))\times Q$ denotes
      a set of transitions. Let $\delta(p,a)=q$ denote every $((p,a),q)\in
      \delta$.
  \end{itemize}
\end{definition}
An NFA also defines a language:
\begin{definition}
  Let $\mathcal{A}=(\Sigma,Q,q_0,F,\delta)$ be an NFA and $L(\mathcal{A})$
  denote the language it accepts, then
  \[
    L(\mathcal{A})=\{w\in\Sigma^*: \exists q_0,q_1,\ldots,q_n\in Q\quad
    \alpha(a_1,a_2,\ldots,a_n,w)\land
    \beta(q_0,a_1,q_1,a_2,q_2,\ldots,a_n,q_n,\delta)\land
    (q_n\in F)\}
  \]
  where
  \begin{itemize}
    \item $\Sigma^*$ denotes all strings over $\Sigma$. In other words,
      \[
        \Sigma^*=\bigcup_{n\in\mathds{N}}
        \{a_1\mathbin\Vert a_2\mathbin\Vert\cdots\mathbin\Vert a_n:
        a_1,a_2,\ldots,a_n\in\Sigma\}
      \]
    \item $\alpha(a_1,a_2,\ldots,a_n,w)$ denotes the following boolean
      expression:
      \[
        a_1\mathbin\Vert a_2\mathbin\Vert\cdots\mathbin\Vert a_n=w
      \]
    \item $\beta(q_0,a_1,q_1,a_2,q_2,\ldots,a_n,q_n,\delta)$ defines the following
      boolean expression:
      \[
        \bigwedge_{i=0,\ldots,n-1}\delta(q_i,a_{i+1})=q_{i+1}
      \]
  \end{itemize}
\end{definition}
The connection between NFA and regex is the following theorem:
\begin{theorem}
  For every regex $\gamma$ over finite alphabet $\Sigma$, there is a NFA
  $\mathcal{A}$ such that $L(\gamma)=L(\mathcal{A})$.
\end{theorem}
\begin{proof}
  We prove this by induction on $\gamma$:
  \begin{itemize}
    \item Base case:
      \begin{itemize}
        \item If $\gamma=\varnothing$, then let $\mathcal{A}=(\Sigma,\{q_0\},
          q_0,\varnothing,\varnothing)$.
        \item If $\gamma=a\in\Sigma$, then let $\mathcal{A}=(\Sigma,Q,q_0,F,
          \delta)$ where
          \begin{itemize}
            \item $Q=\{q_0,q_1\}$.
            \item $F=\{q_1\}$.
            \item $\delta=\{((q_0,a),q_1)\}$.
          \end{itemize}
      \end{itemize}
      In both cases, it is obvious that $L(\alpha)=\{a\}=L(\mathcal{A})$.
    \item Induction case: suppose $\alpha,\beta$ are regexes, $\mathcal{A}_1
      =(\Sigma,Q_1,q_{10},F_1,\delta_1), \mathcal{A}_2=(\Sigma,Q_2,q_{20},F_2,
      \delta_2)$ are NFAs, and $L(\alpha)=L(\mathcal{A}_1),L(\beta)=
      L(\mathcal{A}_2)$ by inductive hypothesis, then
      \begin{itemize}
        \item If $\gamma=\alpha\beta$, then let $\mathcal{A}=(\Sigma,
          Q_1\cup Q_2,q_{10},F_2,\delta)$ where
          \[
            \delta=\delta_1\cup\delta_2\cup((F_1\times\{\varepsilon\})
            \times\{q_{20}\})
          \]
        \item If $\gamma=\alpha|\beta$, then let $\mathcal{A}=(\Sigma,
          Q_1\cup Q_2,q_{10},F_1\cup F_2,\delta)$ where
          \[
            \delta=\delta_1\cup\delta_2\cup\{((q_{10},\varepsilon),q_{20})\}
          \]
        \item If $\gamma=\alpha^*$, then let $\mathcal{A}=(\Sigma,
          Q_1,q_{10},F_1,\delta)$ where
          \[
            \delta=\delta_1\cup((F_1\times\{\varepsilon\})\times\{q_{10}\})
          \]
      \end{itemize}
      In all three cases, it is clear that $L(\gamma)=L(\mathcal{A})$.
  \end{itemize}
\end{proof}
This proof also gives a (recursive) constructive algorithm for building an
NFA that accepts the same language as an arbitrary regex.

\section{Deterministic Finite Automata}
An NFA is not ideal for implementation because:
\begin{itemize}
  \item From a state $q$ reading input $a$, we can go to multiple different
    states.
  \item From a state $q$ without reading any input, we can go to multiple
    different states. These are called $\varepsilon$-transitions.
\end{itemize}
A \textit{deterministic finite automata} solves this issue:
\begin{definition}
  A \textit{deterministic finite automata} (\textit{DFA} for short) is a 5-tuple
  $\mathcal{A}=(\Sigma,Q,q_0,F,\delta)$ where
  \begin{itemize}
    \item $\Sigma$ is a finite alphabet.
    \item $Q$ is a set of states.
    \item $q_0\in Q$ is the initial state.
    \item $F\subseteq Q$ is a set of accepting states.
    \item $\delta:I\to Q$ is a set of transitions where
      $I\subseteq Q\times\Sigma$.
  \end{itemize}
\end{definition}
The only difference between a DFA and NFA is the set of transitions. The
definition of the language accepted by a DFA is also the same as that of an
NFA\@. The connection between a DFA and an NFA is the following theorem:
\begin{theorem}
  For every NFA $\mathcal{A}$, there is a DFA $\mathcal{A}'$ such that
  $L(\mathcal{A})=L(\mathcal{A}')$.
\end{theorem}
We omit the proof, and only show how to construct such DFA\@. We first define
what an \textit{$\varepsilon$-closure} is:
\begin{definition}
  Let $\mathcal{A}=(\Sigma,Q,q_0,F,\delta)$ be an NFA, and let $q\in Q$. The
  \textit{$\varepsilon$-closure of $q$} is a set
  \[
    \{q\}\cup\{p\in Q:\delta(q,\varepsilon)=p\}
  \]
  we typically denote such a set as $\varepsilon\text{-closure}(q)$.
\end{definition}
Then, the following is the algorithm to construct the DFA $\mathcal{A}'$ from an
NFA $\mathcal{A}$:
\begin{breakablealgorithm}
  \caption{\textproc{NFA To DFA Conversion}}
  \begin{algorithmic}[1]
    \Function{NFA-To-DFA-Conversion}{$\mathcal{A}=(\Sigma,Q,q_0,F,\delta)$}
      \State{$Q'\leftarrow\varnothing$}
      \For{$q\in Q$}
        \State{$Q'\leftarrow Q'\cup\varepsilon\text{-closure}(q)$}
        \MyComment{each state of the DFA is a subset of $Q$}
      \EndFor{}
      \State{$q_0'\leftarrow\varepsilon\text{-closure}(q_0)$}
      \State{$F'\leftarrow\varnothing$}
      \For{$q'\in Q'$}
        \If{$q'\cap F\neq\varnothing$}
          \State{$F'\leftarrow F'\cup\{q'\}$}
          \MyComment{any DFA state containing an accepting NFA state is an
                     accepting DFA state}
        \EndIf{}
      \EndFor{}
      \State{$\delta'\leftarrow\varnothing$}
      \For{$p'\in Q'$}
        \For{$a\in\Sigma$}
          \State{$q'\leftarrow\varnothing$}
          \For{$p\in p'$}
            \For{$((p,a),q)\in\delta$}
              \State{$q'\leftarrow q'\cup\varepsilon\text{-closure}(q)$}
              \MyComment{combine all reachable NFA states (when going from an NFA
                         state $p\in p'$ and reading $a$) into a DFA state $q'$}
            \EndFor{}
          \EndFor{}
        \EndFor{}
      \EndFor{}
      \State{$\mathcal{A}'\leftarrow(\Sigma,Q',q_0',F',\delta')$}
      \State{\Return{$\mathcal{A}'$}}
    \EndFunction{}
  \end{algorithmic}
\end{breakablealgorithm}

\section{DFA Minimization}
For efficent lexing, we would want the minimal DFA that accepts the same
language. The following theorem applies to all DFA\@:
\begin{theorem}
  For every DFA $\mathcal{A}$, there is a unique minimal DFA
  $\mathcal{A}'=(\Sigma,Q',q_0',F',\delta')$ such that
  \begin{itemize}
    \item $L(\mathcal{A})=L(\mathcal{A}')$.
    \item $|Q'|$ is minimized.
  \end{itemize}
\end{theorem}
This proof is also omitted. We provide the algorithm to minimize such DFA\@:
\begin{breakablealgorithm}
  \caption{\textproc{DFA Minimization}}
  \begin{algorithmic}[1]
    \Function{DFA-Minimization}{$\mathcal{A}=(\Sigma,Q,q_0,F,\delta)$}
      \State{$P\leftarrow\{F,Q\setminus F\}$}
      \MyComment{partition the original states $Q$, each partition represents
                 a set of potentially indistinguishable states}
      \State{$\text{changed}\leftarrow\textbf{true}$}
      \While{changed}
        \State{$\text{changed}\leftarrow\textbf{false}$}
        \For{$p'\in P$}
          \State{$\text{comesFrom}\leftarrow\text{an empty dictionary}$}
          \For{$a\in\Sigma$}
            \For{$p\in p'$}
              \State{$q'\leftarrow q'\in P$ such that $\delta(p,a)\in q'$}
              \MyComment{check which partition $q'$ each state $p$ in the
                         current partition $p'$ goes to when reading $a$}
              \If{$\text{comesFrom}[q']$ is set}
                \State{$\text{comesFrom}[q']\leftarrow\text{comesFrom}[q']
                       \cup\{p\}$}
              \Else{}
                \State{$\text{comesFrom}[q']\leftarrow\{p\}$}
              \EndIf{}
            \EndFor{}
            \If{$|\text{comesFrom.keys}|\neq1$}
            \MyComment{states in current partition $P'$ goes to different
                       partitions, thus we must change the current partition}
              \State{$P\leftarrow(P\setminus\{p'\})\cup\text{comesFrom.values}$}
              \State{$\text{changed}\leftarrow\textbf{true}$}
              \State{\textbf{break}}
            \EndIf{}
          \EndFor{}
          \If{changed}
            \State{\textbf{break}}
          \EndIf{}
        \EndFor{}
      \EndWhile{}
      \State{$Q'\leftarrow\varnothing$}
      \State{$F'\leftarrow\varnothing$}
      \For{$p'\in P$}
        \If{$p'=\varnothing$}
          \State{\textbf{continue}}
        \EndIf{}
        \If{$q_0\in p'$}
          \State{$p\leftarrow q_0$}
          \MyComment{make sure to choose $q_0$}
        \Else{}
          \State{$p\leftarrow$ any $p\in p'$}
          \MyComment{otherwise, choosing any state from the partition $p'$ is
                     sufficient because they are indistinguishable}
        \EndIf{}
        \State{$Q'\leftarrow Q'\cup\{p\}$}
        \If{$p\in F$}
          \State{$F'\leftarrow F'\cup\{p\}$}
        \EndIf{}
      \EndFor{}
      \State{$\delta'\leftarrow\varnothing$}
      \For{$((p,a),q)\in\delta$}
        \If{$p\in Q'$ and $q\in Q'$}
          \State{$\delta'\leftarrow\delta'\cup\{((p,a),q)\}$}
        \EndIf{}
      \EndFor{}
      \State{$\mathcal{A}\leftarrow(\Sigma,Q',q_0,F',\delta')$}
      \State{\Return{$\mathcal{A}$}}
    \EndFunction{}
  \end{algorithmic}
\end{breakablealgorithm}

\section{Multiple Tokens}
So far we have only considered building a DFA for one token. To build a DFA
for multiple tokens, we do the following steps:
\begin{enumerate}
  \item Build the regex for each token.
  \item Convert the regexes to NFAs.
  \item Combine the NFAs into one big NFA using $\varepsilon$-transitions.
  \item Convert the big NFA into a DFA\@.
  \item Minimize the DFA\@.
\end{enumerate}
The only tricky part is minimizing the DFA\@. To be able to identify which
type of token is read, we must initially split the accepting states of the DFA
into different partitions if the accepting states it contains from different
the original NFAs. Furthermore, because each accepting state in the DFA can now
contain accepting states from different NFAs, choosing when to accept and which
kind of token to accept will depend on
\begin{itemize}
  \item Longest matching rule: match the longest token; this will require
    backtracking.
  \item User preference: define a priority on which tokens to be accepted first.
  \item Whitespace: typically, a programming language contains separating
    characters such as whitespace, and reading such character tells us to
    accept now.
\end{itemize}

\section{Tokens Defined By Fartlang}

\end{document}
