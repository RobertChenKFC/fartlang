\documentclass[titlepage]{article}
\usepackage{amssymb}
\usepackage[
	margin=1.5cm,
	includefoot,
	footskip=30pt,
]{geometry}

\usepackage{amsthm}
\usepackage[open,openlevel=0]{bookmark}
\usepackage{mathrsfs}
\usepackage{dsfont}
\usepackage{amsmath}
\usepackage[dvipsnames]{xcolor}
\usepackage{tikz}
\usepackage{multirow}
\usepackage{CJKutf8}
\usepackage{enumerate}
\usepackage{minted}
\usepackage{listings}
\usepackage{xcolor}
\usepackage{wasysym}
\usepackage{enumerate}
\usepackage[RPvoltages]{circuitikz}
\usepackage{soul}
\usepackage{karnaugh-map}
\usepackage{booktabs}
\usepackage{subcaption}
\usepackage{algorithm}
\usepackage{algpseudocode}
\newcommand{\pushcode}[1][1]{\hskip\dimexpr#1\algorithmicindent\relax}
\newcommand{\MyComment}[2][.5] {\Comment{\parbox[t]{#1\linewidth}{#2}}}
\makeatletter

\newenvironment{breakablealgorithm}
  {% \begin{breakablealgorithm}
   \begin{center}
     \refstepcounter{algorithm}% New algorithm
     \hrule height.8pt depth0pt \kern2pt% \@fs@pre for \@fs@ruled
     \renewcommand{\caption}[2][\relax]{% Make a new \caption
       {\raggedright\textbf{\ALG@name~\thealgorithm} ##2\par}%
       \ifx\relax##1\relax % #1 is \relax
         \addcontentsline{loa}{algorithm}{\protect\numberline{\thealgorithm}##2}%
       \else % #1 is not \relax
         \addcontentsline{loa}{algorithm}{\protect\numberline{\thealgorithm}##1}%
       \fi
       \kern2pt\hrule\kern2pt
     }
  }{% \end{breakablealgorithm}
     \kern2pt\hrule\relax% \@fs@post for \@fs@ruled
   \end{center}
  }
\makeatother

\usetikzlibrary{arrows, shapes.gates.logic.US, calc, positioning}

\theoremstyle{plain}
\newtheorem*{axiom}{Axiom}
\newtheorem*{theorem}{Theorem}
\newtheorem*{observation}{Observation}
\newtheorem*{exercise}{Exercise}

\theoremstyle{definition}
\newtheorem*{definition}{Definition}

\theoremstyle{remark}
\newtheorem*{solution}{Solution}
\newtheorem*{remark}{Remark}

\DeclareMathOperator{\dom}{dom}

\title{Syntax Analysis}
\author{Robert Chen}

\begin{document}

\maketitle

\newpage
\tableofcontents

\newpage
\section{Context Free Grammar}
The following is the definition of a \textit{context free grammar}:
\begin{definition}
  A \textit{context free grammar} (\textit{CFG} for short) is a 4-tuple
  $\mathcal{G}=(\Sigma,V,R,S)$ where
  \begin{itemize}
    \item $\Sigma$ is a finite alphabet.
    \item $V$ is a set of variables.
    \item $R$ is a set of rules, where each rule $r$ is in the form $r: A
      \rightarrow w$, with $A\in V$ and $w\in {(V\cup\Sigma)}^*$.

      We denote $A$ as the \textit{left-hand-side} (\textit{LHS}) and $w$ as the
      \textit{right-hand-side} (\textit{RHS}).
    \item $S\in V$: the starting variable.
  \end{itemize}
\end{definition}
A CFG also defines a language. First let's see the definition of a
\textit{derivation}:
\begin{definition}
  Let $\mathcal{G}=(\Sigma,V,R,S)$ be a CFG, $u=u_1Au_2$ where
  $u_1,u_2\in{(V\cup\Sigma)}^*$ and $A\in V$, and $r$ be a rule
  $r: A\rightarrow w$ in $R$. A \textit{derivation} on $u$ is denoted as
  \[
    u_1Au_2\xrightarrow{r}u_1wu_2
  \]
  which represents changing a variable in $u$ to the RHS of rule $r$.
\end{definition}
Then the definition of the language of a CFG follows:
\begin{definition}
  Let $\mathcal{G}=(\Sigma,V,R,S)$ be a CFG and $L(\mathcal{G})$ denote the
  language that $\mathcal{G}$ accepts, then
  \[
    L(\mathcal{G})=\{w\in\Sigma^*:\exists\text{ a finite sequence of derivations
    from }S\text{ to }w\}
  \]
\end{definition}

\section{Determinism and Ambiguity}
\begin{remark}
  This section is mainly to provide some context, so most of the content is
  not complete. Many interesting
  theoretical results are not included, as they are not the focus of this
  documentation.
\end{remark}
Similar to regexes, CFGs also have a class of machines that accept the languages
of CFGs, called the \textit{push down automaton} (\textit{PDA} for short).
Similar to finite automata, PDAs also have non-deterministic and deterministic
(denoted as \textit{DPDA}) variants. Unfortunately, DPDAs are not as powerful as
general PDAs, meaning there are CFGs that cannot be parsed by DPDAs but can be
parsed by PDAs. Thus, the notion of \textit{deterministic CFGs} come in:
\begin{definition}
  A CFG $\mathcal{G}$ is \textit{deterministic} there is a corresponding DPDA
  that derives $\mathcal{G}$.
\end{definition}
Deterministic CFGs (DCFG for short) are often preferred because the DPDAs that
derive them yields better run-time performance (similar idea to DFA).
Another property of CFG is its \textit{ambiguity}. For any CFG $\mathcal{G}
=(\Sigma,V,R,S)$, every $w\in L(\mathcal{G})$ can be obtained from a finite
sequence of derivations on $S$. Even if we limit the sequence of derivations to
derive the leftmost variable first, the language $L(\mathcal{G})$ is still
unchanged. This kind of derivation is called a \textit{leftmost-derivation}. The
definition of the \textit{ambiguity} of a CFG follows:
\begin{definition}
  A CFG $\mathcal{G}=(\Sigma,V,R,S)$ is \textit{ambiguous} if there is a $w\in
  L(\mathcal{G})$ that can obtained from two different finite sequences of
  leftmost derivations on $S$.
\end{definition}
Ambiguous CFGs are also something we want to avoid, since it may result in a
program that can be parsed in two different ways, and thus how the program
should be executed becomes unclear. Fortunately, DCFGs are always
unambiguous (however, the converse is not true), thus if we restrict the CFGs
to be deterministic, we don't have to worry about ambiguity.

Therefore, we now focus solely on how to parse DCFGs. We introduce the LR$(1)$
and LALR$(1)$ parsing tables, which are driven by the parsing algorithm to produce
parsers. The LR$(1)$ parser comes from a family of LR$(k)$ parsers, which are
a set of DPDAs. Theoretical results show that the family of LR$(k)$ parsers
are actually \textbf{exactly} the set of DPDAs, and that all LR$(k)$ parsers
can be transformed into LR$(1)$ parsers. In other words, LR$(1)$ parsers can parse all
DCFGs. LALR$(1)$ is a method to reduce the number of states produced by LR$(1)$
parsers.

\section{LR$(1)$ Parsers}
Before getting into the definition of a LR$(1)$ parser, we have to first look
at a few definitions. The following definition all uses the DCFG $\mathcal{G}
=(\Sigma,V,R,S)$.
\begin{definition}
  An \textit{LR$(1)$ item} is a 3-tuple $w'=(r,i,z)$ where
  \begin{itemize}
    \item $r$ is a rule in $R$.
    \item $w\in{(\Sigma\cup V\cup\{\cdot\})}^*$ comes from the RHS of $r$
      with a $\cdot$ inserted at any position.
    \item $z\subseteq\Sigma$ is a set of lookahead symbols.
  \end{itemize}
\end{definition}
\begin{definition}
  For any variable $A\in V$, $A$ is called \textit{nullable} if one of the
  following is satisfied:
  \begin{itemize}
    \item There is a rule $A\rightarrow\varepsilon$ in $R$.
    \item There is a rule $A\rightarrow A_1A_2\cdots A_n$ in $R$, and for every
      $i=1,\ldots,n$, $A_i\in V$ and $A_i$ is nullable.
  \end{itemize}
\end{definition}
\begin{definition}
  The \textit{first-set} of a string $w\in{(\Sigma\cup V)}^*$, denoted by
  $\text{first}(w)$, is a set $z\subseteq\Sigma$ constructed as follows:
  \begin{itemize}
    \item If $w=aw_1$ where $a\in\Sigma$ and $w_1\in{(\Sigma\cup V)}^*$, then
      $a\in z$.
    \item If $w=Aw_1$ where $A\in V$ and $w_1\in{(\Sigma\cup V)}^*$, then
      for every rule $r: A\rightarrow x$, we have $\text{first}(x)\subseteq
      z$.
    \item If $w=Aw_1$ where $A\in V$ and $w_1\in{(\Sigma\cup V)}^*$ and $A$
        is nullable, then $\text{first}(w_1)\subseteq z$.
    \item If $w=\varepsilon$, then $z=\varnothing$.
  \end{itemize}
\end{definition}
\begin{definition}
  Given $w\in{(\Sigma\cup V)}^*$ and $z\subseteq\Sigma$, the
  \textit{follow-set} of $w$ and $z$, denoted by $\text{follow}(w,z)$, is
  defined as
  \[
    \text{follow}(w,z)=\begin{cases}
      z & \text{if }w=\varepsilon \\
      \text{first}(w)\cup z & \text{if }w=Aw_1\text{ where }A\in V\text{ and }
      A\text{ is nullable} \\
      \text{first}(w) & \text{otherwise}
    \end{cases}
  \]
\end{definition}
\begin{definition}
  A \textit{closure} of an LR$(1)$ item $w'$, denoted by $\text{closure}(w')$,
  is a set $q'$ of LR$(1)$ items where a set $q$ is first constructed as
  follows:
  \begin{itemize}
    \item $w'\in q$.
    \item If $(r,w_1\cdot Bw_2,z)\in q$, where $w_1,w_2\in{(\Sigma\cup
      V)}^*$ and $B\in V$, then for every rule $(s:B\rightarrow x)\in R$, we
      have
      \[
        (s,\cdot x,\text{follow}(w_2,z))\in q
      \]
  \end{itemize}
  Then, the final set $q'$ is constructed from $q$ with the following:
  \[
    q'=\left\{(r,w,z):(r,w,z_1),\ldots,(r,w,z_n)\in q\text{ and }
    z=\bigcup_{i=1}^n z_i\right\}
  \]
\end{definition}
With these definitions, we can construct the \textit{LR$(1)$ state transition
graph} using the following algorithm:
\begin{breakablealgorithm}
  \caption{\textproc{LR$(1)$-State-Transiton-Graph}}
  \begin{algorithmic}[1]
    \Function{LR$(1)$-State-Transition-Graph}{}
      \State{$\Sigma\leftarrow\Sigma\cup\{\$\}$}
     \MyComment{$\$$ represents the end-of-string symbol}
      \State{$V\leftarrow V\cup\{S'\}$}
      \State{$r\leftarrow(S'\rightarrow S)$}
      \State{$R\leftarrow R\cup\{r\}$}
      \State{$q_0\leftarrow\text{closure}((r,\cdot S,\$))$}
      \MyComment{The initial state $q_0$ represents the state where it is
        expecting to derive $S$, followed by lookahead symbol $\$$}
      \State{$Q\leftarrow\{q_0\}$}
      \State{$\delta\leftarrow\varnothing$}
      \For{$p\in Q$}
        \For{$x\in\Sigma\cup V$}
          \State{$q\leftarrow\varnothing$}
          \For{$(r,w_1\cdot w_2,z)\in p$}
            \If{$w_2=xw_3$}
              \State{$q\leftarrow\text{closure}((r,w_1x\cdot w_3,z))$}
            \EndIf{}
          \EndFor{}
          \State{$Q\leftarrow Q\cup\{q\}$}
          \State{$\delta\leftarrow\delta\cup\{((p,x),q)\}$}
        \EndFor{}
      \EndFor{}
      \State{\Return{$(Q,q_0,\delta)$}}
    \EndFunction{}
  \end{algorithmic}
\end{breakablealgorithm}
After constructing the state-transition graph, we can use it to construct the
\textit{LR$(1)$ state table} using the following algorithm:
\begin{breakablealgorithm}
  \caption{\textproc{LR$(1)$-State-Table}}
  \begin{algorithmic}[1]
    \Function{LR$(1)$-State-Table}{$Q,q_0,\delta$}
      \State{$T\leftarrow $ a two-dimensional array, all initialized to
             error}
      \For{$((p,x),q)\in\delta$}
        \If{$x\in\Sigma$}
          \State{$T[p][x]\leftarrow\text{shift}(q)$}
          \MyComment{Represents a \textit{shift} action}
        \Else{}
          \State{$T[p][x]\leftarrow\text{goto}(q)$}
          \MyComment{Represents a \textit{goto} action}
        \EndIf{}
      \EndFor{}
      \For{$p\in Q$}
        \For{$(r,w_1\cdot w_2,z)\in p$}
          \If{$w_2=\varepsilon$}
            \For{$a\in z$}
              \If{$T[p][a]\neq\text{error}$ and $T[p][a]\neq\text{reduce}(r)$}
                \If{$T[p][a]$ is a shift action}
                  \State{\textbf{print}("shift-reduce conflict on state $p$")} % chktex 18 chktex 36
                \ElsIf{$T[p][a]$ is a reduce action}
                  \State{\textbf{print}("reduce-reduce conflict on state $p$")} % chktex 18 chktex 36
                \EndIf{}
                \State{\Return{\textbf{null}}}
              \Else{}
                \State{$T[p][a]\leftarrow\text{reduce}(r)$}
                \MyComment{Represents a \textit{reduce} action}
              \EndIf{}
            \EndFor{}
          \EndIf{}
        \EndFor{}
      \EndFor{}
      \State{\Return{$(q_0,T)$}}
    \EndFunction{}
  \end{algorithmic}
\end{breakablealgorithm}
Note that we introduced the concept of \textit{conflicts} --- states where
different actions can be taken. If conflicts exist, then it is not possible to
build an LR$(1)$ parser using the above algorithm.

With the LR$(1)$ state table constructed, we can finally run the parsing
algorithm on any word $s\in\Sigma^*$:
\begin{breakablealgorithm}
  \caption{\textproc{Parsing Algorithm}}
  \begin{algorithmic}[1]
    \Function{Parsing-Algorithm}{$q_0,T,s$}
      \State{$s\leftarrow s\$$}
      \State{$Z\leftarrow$ an empty stack}
      \State{$Z$.push($q_0$)} % chktex 36
      \State{$p\leftarrow q_0$}
      \For{each character $a$ of $s$ in order}
        \State{$\text{action}\leftarrow T[p][a]$}
        \If{$\text{action}=\text{error}$}
          \State{\textbf{print}("Syntax error")} % chktex 18 chktex 36
          \State{\Return{\textbf{false}}}
        \ElsIf{$\text{action}=\text{shift}(q)$ for some $q\in Q$}
          \State{$Z$.push($a$)} % chktex 36
          \State{$Z$.push($q$)} % chktex 36
          \State{$p\leftarrow q$}
        \ElsIf{$\text{action}=\text{reduce}(r)$ for some $r\in R$}
          \State{$n\leftarrow$ the length of the RHS of $r$}
          \For{$2n$ times}
            \State{$Z$.pop()} % chktex 36
          \EndFor{}
          \State{$p\leftarrow Z$.top()} % chktex 36
          \State{$A\leftarrow$ LHS of $r$}
          \State{$q\leftarrow T[p][A]$} % chktex 36
          \State{$Z$.push($a$)} % chktex 36
          \State{$Z$.push($q$)} % chktex 36
          \State{$p\leftarrow q$}
        \EndIf{}
      \EndFor{}
      \State{\Return{\textbf{true}}}
    \EndFunction{}
  \end{algorithmic}
\end{breakablealgorithm}

\section{LALR$(1)$ Parsers}
LALR$(1)$ parsers are exactly the same as LR$(1)$ parsers except for the
construction of the state table. The following algorithm shows how to construct
an LALR$(1)$ state table from an LR$(1)$ state table:
\begin{breakablealgorithm}
  \caption{\textproc{LALR$(1)$ State Table}}
  \begin{algorithmic}[1]
    \Function{LALR$(1)$-State-Table}{$q_0,T$}
      \State{$\text{newItems}\leftarrow\text{an empty dictionary}$}
      \For{$q\in Q$}
        \State{$W\leftarrow\varnothing$}
        \For{$(w,z)\in q$}
          \State{$W\leftarrow W\cup\{w\}$}
        \EndFor{}
        \If{$q=q_0$}
          \State{$W_0\leftarrow W$}
        \EndIf{}
        \If{$\text{newItems}[W]$ exists}
          \State{$\text{newItems}[W]\leftarrow\text{newItems}[W]\cup\{q\}$}
        \Else{}
          \State{$\text{newItems}[W]\leftarrow\{q\}$}
        \EndIf{}
      \EndFor{}
      \State{Find $q_0'$ such that $q_0\in\text{newItems}[q_0']$}
      \State{$T'\leftarrow$ a two-dimensional array, all initialized to
             error}
      \For{$p'\in\text{newItems.values}$}
        \For{$p\in p'$}
          \For{$x\in\Sigma\cup V$}
            \State{$\text{action}_\text{new}\leftarrow T[p][x]$}
            \If{$\text{action}_\text{new}=\text{shift}(q)$ for some $q\in Q$}
              \State{Find $q'$ such that $q\in\text{newItems}[q']$}
              \State{$\text{action}_\text{new}\leftarrow\text{shift}(q')$}
            \ElsIf{$\text{action}_\text{new}=\text{goto}(q)$ for some $q\in Q$}
              \State{Find $q'$ such that $q\in\text{newItems}[q']$}
              \State{$\text{action}_\text{new}\leftarrow\text{goto}(q')$}
            \EndIf{}
            \If{$T'[p'][x]\neq\text{error}$ and
                $\text{action}_\text{new}\neq\text{error}$ and
                $T'[p'][x]\neq\text{action}_\text{new}$}
              \If{$\text{action}_\text{new}$ is a shift action}
                \State{\textbf{print}("shift-reduce conflict on state $p'$")} % chktex 18 chktex 36
              \ElsIf{$\text{action}_\text{new}$ is a reduce action}
                \State{\textbf{print}("reduce-reduce conflict on state $p'$")} % chktex 18 chktex 36
              \EndIf{}
              \State{\Return{\textbf{null}}}
            \EndIf{}
            \State{$T'[p'][x]\leftarrow\text{action}_\text{new}$}
          \EndFor{}
        \EndFor{}
      \EndFor{}
      \State{\Return{$(q_0',T')$}}
    \EndFunction{}
  \end{algorithmic}
\end{breakablealgorithm}
Note that the construction of the LALR$(1)$ parsing table may introduce new
conflicts, which means that even if a CFG can be used to construct an LR$(1)$
parser, the same CFG may not necessarily be able to be used to construct an
LALR$(1)$ parser.

\section{Empty Productions}
The algorithms above only work when all rules in the CFG are not empty productions;
in other words, all $r:A\rightarrow x$ in $R$ satisfies $x\neq\varepsilon$.
However, having empty productions (i.e. $r:A\rightarrow\varepsilon$) can be
very helpful in constructing the CFG for various programming language
constructs. Therefore, here we show how to remove empty productions from a
given CFG\@:
\begin{breakablealgorithm}
  \caption{\textproc{Empty Production Removal}}
  \begin{algorithmic}[1]
    \Function{Empty-Production-Removal}{$\mathcal{G}=(\Sigma,V,R,S)$}
      \State{$\text{changed}\leftarrow\textbf{true}$}
      \While{changed}
        \State{$\text{changed}\leftarrow\textbf{false}$}
        \For{$(r:A\rightarrow x)\in R$}
          \If{$x=\varepsilon$}
            \For{$(s:B\rightarrow y)\in R$}
              \State{$n\leftarrow$ the number of occurences of $A$ in $y$}
              \If{$n>0$}
                \For{$m=0,1,\ldots,2^n-1$}
                  \State{$y'\leftarrow y$ with the $i$-th occurrence of $A$
                    removed if the $i$-th bit of $m$ is 1}
                  \State{$R\leftarrow R\cup\{B\rightarrow y'\}$}
                  \State{$\text{changed}\leftarrow\textbf{true}$}
                \EndFor{}
              \EndIf{}
            \EndFor{}
          \EndIf{}
        \EndFor{}
      \EndWhile{}
    \EndFunction{}
  \end{algorithmic}
\end{breakablealgorithm}

\end{document}
